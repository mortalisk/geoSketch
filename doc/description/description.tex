\documentclass[12pt,a4paper]{article}
\usepackage{graphicx}

\begin{document}

\title{Description of master thesis}
\author{Morten Bendiksen}
\date{Mars 2011}
\maketitle

\section*{Rapid modeling of geological structures}


Geologist often make sketches of geological structures, both in order to communicate ideas amongst themselves, and to other interested people. We propose to develop a computer program to aid in this sketching.

In developing this program, the following techniques will be explored:
\begin{itemize}
 \item Having an initial empty sandbox from which the structures can be ``carved''
 \item Drawing layers by turning and sketching on the sides of the box
    \begin{itemize}
     \item Layers will be interpolated from this
     \item Modifying layers by sketching on them and pushing or pulling
    \end{itemize}
 \item Drawing rivers by sketching on the surface of horizons
    \begin{itemize}
     \item Will carve out a plausible river following this path
     \item Allows adjustments of size and depth
    \end{itemize}
\item Picking layers with mouse pointer
    \begin{itemize}
     \item Further editing of layer is then possible
     \item Changing color
     \item Setting tranparency
    \end{itemize}
\end{itemize}


\section*{Illustrations of use case}
Here we show a possible sequence of manipulations to  quickly create some geological structures in a scene.
\begin{figure}
\centering
\includegraphics[width=4in]{box}
\caption[]{
  \footnotesize
  We start with the empty box
  \label{fig:box}
}
\end{figure}

\begin{figure}
\centering
\includegraphics[width=4in]{turnBox}
\caption[]{
  \footnotesize
  We draw the imagined layer in the box by turning it and drawing on the sides
  \label{fig:turnBox}
}
\end{figure}

\begin{figure}
\centering
\includegraphics[width=4in]{interpolateLayer}
\caption[]{
  \footnotesize
  A layer is interpolated from the four sides we draw
  \label{fig:interpolateLayer}
}
\end{figure}

\begin{figure}
\centering
\includegraphics[width=4in]{drawRiver}
\caption[]{
  \footnotesize
  We draw a river path on this layer
  \label{fig:drawRiver}
}
\end{figure}

\begin{figure}
\centering
\includegraphics[width=4in]{carveRiver}
\caption[]{
  \footnotesize
  The computer will carve out from this layer as needed to make a river follow this path in a plausible way
  \label{fig:carveRiver}
}
\end{figure}

\begin{figure}
\centering
\includegraphics[width=4in]{drawNewLayer}
\caption[]{
  \footnotesize
  Now we draw a new layer. This can use the previous layer as a drawing surface in stead of only the sides of the box
  \label{fig:drawNewLayer}
}
\end{figure}

\begin{figure}
\centering
\includegraphics[width=4in]{addColor}
\caption[]{
  \footnotesize
  We add some color to the layers. In this figure the layers are partially transparent.
  \label{fig:addColor}
}
\end{figure}

\begin{figure}
\centering
\includegraphics[width=4in]{offTransparency}
\caption[]{
  \footnotesize
  Here we have turned of tranparency and the sides become opaque
  \label{fig:offTransparency}
}
\end{figure}

\end{document}

