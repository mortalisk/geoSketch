\documentclass[a4paper,10pt]{article}
\usepackage[utf8]{inputenc}

%opening
\title{Plan}
\author{Morten Bendiksen}

\begin{document}

\maketitle

\begin{abstract}

\end{abstract}

\section*{DONE}
Features that are working
\subsection*{Draw Box}
Having an initial empty box on which the structures can be sketched
\subsection*{Input raw spline points}
Captures the mouse input from the user on the box or other surface.

\section*{TODO}
These are features that are not done. I tried to put them in order of importance.
\subsection*{Estimate time}
Estimate time for the features that are neccessary in this list.
\subsection*{Cube visualization}
The cube visualization must be improved. There are several ways to do this, will start with the simplest one.
\begin{itemize}
 \item The edges of the cube can be drawn.
 \item There could be a grid on the cube.
 \item There could be shadows on the cube.
\end{itemize}
\subsection*{Create parameterization}
Create a parameterized representation of a surface. This is partially done, but in a half working, stupid way.
\subsection*{Edit surface}
Must be able to pick an existing horizon to edit. Find good user friendly technique for this.
\subsection*{Drawing modes}
Need some way of switching between modes of editing. I see these modes as being useful. How to switch between them is another thing to think about.
\begin{itemize}
 \item Make surface
 \item Edit surface
 \item Sketch feature
 \item Edit feature
\end{itemize}
\subsection*{Oversketching}
The ability to easily modify the sketch by drawing strokes that go outside the lines. 
\subsection*{Shading}
There must be lighting and shading on the surfaces, in order to better visualize their topology.
\subsection*{Delete}
To be able to undo/redo
\subsection*{Spline order}
There should be an enforced order to draw the splines, or to interpret the splines. Otherwize the result can be unexpected.
\subsection*{Profile drawing}
To draw the profile of features on the terrain on seperate canvas
\subsection*{Prepare for meeting with geologists}
For the meeting I want to get as much useful information as possible. In order for this to be possible, there should already be a first ``story''. I want to show them a possible scenario that is now possible with the program, and what I see as possible goals/directions. If they ``get it'' and are able to contribute with their vision for completing the story, this would be very useful.
\subsection*{Procedural linking}
To be able to model the start scenario witht for example a river, and them have the program modify the terrain to create deposits from the river, is a useful feature. A possibility to also go in the other direction, and demand some deposits in an area, and then have the program create a scaenario with a river, could be interessting.
\subsection*{Drawing on the surfaces}
To sketch features on the surfaces, to either modify them directly, or direct a procedural editor/creator.
\subsection*{Brush size}
Brush size modifies the accuracy of strokes

\end{document}


  