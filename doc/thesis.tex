\documentclass[a4paper,10pt]{article}
\usepackage[utf8]{inputenc}
\usepackage{cite}

%opening
\title{Rapid Geologic Modeling}
\author{Morten Bendiksen}

\begin{document}

\maketitle

\begin{abstract}
Describe thesis
\end{abstract}

\newpage

\tableofcontents 

\newpage

% --- OBS, gammel tekst i git commit 53adef7c5ef8730ab96a392efa1a10f1ae289f0b ---

\section{Introduction}

(present problem)\\
	- sketch 2d models\\
	- need tools for 3d modelling\\
	- existing 3d tools are advanced\\
	- need a simple/intuitive way to model\\
Illustrations are important for geologists and their aspiring students when they are conducting their everyday business. It is normal to make sketched models by hand on either paper or computer. These are used in both professional and educational settings, and facilitate communication. On paper one is of course limited to making 2D models. This can be limiting, since the fenomena of geology are of course 3D. One can sketch 3D fenomena by using perspective drawing techniques, but the model is still confined to the 2D nature of the medium.

On the computer it is possible to make 3D models. However, existing tools are often complex, aimed at creating advanced and detailed models, and requires training to understand and use. It would therefore be nice to have a possibility to make quick 3D models for illustrative purposes in a simple and intuitive way.


(present goal)\\
	- replacement of napkin and pen\\
	- not accurate/detailed\\
	- but illustrative of ideas\\
	- for surveys or education\\
That is what I am trying to create a foundation for in this project. The goal is to make a replacement for pen and paper 2D models. In other words not for situations where a high degree of accuracy is needed and/or attainable, but rather when the goal is to illustrate ideas and communicate them. I foresee a use for this in initial stages of surveys, in education, for communication and in books about geology.

(solution/approach)\\
	- the box metaphor
	- 
It is in books and papers from the geologic field I have found the most inspiration for this project. One way geological phenomena are often illustrated in such situations is what I call the ``box-model''. Examples of such illustrations can be seen in section \ref{sec:geology}, for example in figure (INSERT REF TO FIGURE). The illustration is drawn inside a box cutout of the area of interest. This gives a good way to illustrate the spatial relation of elements in the model. Such illustrations gave rise to the approach I have choosen. 
	
In order to understand the solution it is neccessary to introduce some relevant geologic terminology and background. This introduction can be found in the following section (section \ref{sec:geology}).

\section{Geologic Background}
\label{sec:geology}
- introduce gelogic terms
\subsection{Structured geology}
\subsection{Sedimentary geology}
\subsection{Rocks}
\subsection{Layers}
\subsection{Faults}
\subsection{Rivers}
\subsection{Time}


\section{State of the Art}
\subsection{Modeling}
- fundamental part of reasoning\\
- isolates relevant info\\
- usefulness\\
  -- users understanding of relation to reality\\
- examples (equation, wood models, etc.)

\subsection{Geologic Modeling}
- models subsurface of the earth.\\
- show examples\\(clay, wood, sand, 2d skethces and drawings, 3d computer models)

\subsection{Computer Modeling}
- great power to be exploited\\
- examples (spreadsheet, 3d models, simulations...)\\
\subsubsection{Rapid Computer Modeling}
- intuitive/quick (to learn or use)\\
- replacement for paper\\
  -- gives extra advantages over paper( rotate, change, collaborate, compute )\\
- explain different approaches (sketch based, procedural)\\
- show examples(teddy, z-brush, etc..)\\


\section{Basic Concept}
\subsection{Overview}
- Give an overview of solution(illustration)\\
- Visual model of solution
  -- idea/input/feedback/modification/
\subsection{Features}
- relate back to the geologic background\\
- how they will be achieved\\
   -- Time\\
   -- Layers\\
   -- etc.\\

\section{Methodology}
- how the algorithms works
\subsection{Sketch methods and interpretation}
- intersection\\
- points are gathered\\
- remove noise\\
- auto-complete
\subsection{Representation}
- lines\\
- surfaces\\
- rivers\\
- mountains\\
- etc.
\subsection{Geometry synthesis}
- layers\\
- mountains\\
- rivers\\
- sediments??
\subsection{Rendering}
- surfaces\\
- rivers and overlapping\\
- mountains

\section{Results and performance}
\subsection{Produced Renderings}
- compare with goal\\
- Marie might draw her stuff?
\subsection{Performance of user}
- learning\\
- using
\subsection{Performance of program}
- rendering speed on different hardware

\section{Evaluation}
 -- (focus on expressiveness) -- \\
- How well problem was solved\\
- Where it will be used \\
- Discussion

\section{Conclusion}
- Give short summary\\
- Further work\\
\\


Testing references: Cherlin ~\cite{Cherlin:2005:SMF:1090122.1090145} explores techniques for rapid sketch based 3d modeling.


\bibliography{thesis}{}
\bibliographystyle{plain}
\end{document}
