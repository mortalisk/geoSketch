\documentclass[a4paper,10pt]{article}
\usepackage[utf8]{inputenc}
\usepackage{cite}

%opening
\title{Rapid Geologic Modeling}
\author{Morten Bendiksen}

\begin{document}

\maketitle

\begin{abstract}
This thesis investigates new techniques for rapid modeling of geological structures. It uses sketch-based and procedural-based techniques for making quick sketches that can be used for communication amongst geologists and from geologists to lay persons. The effectiveness of these techniques are discussed in collaboration with geologist.
\end{abstract}

\section{Introduction}
When geologists are surveying a new area, they use sketches to communicate with each other and with decision makers. To facilitate this process I propose that a computer assisted sketch based modeling approach would increase productivity and enhance communication.

\subsection{Geologic modeling}

\subsection{Sketch based modeling}
\subsubsection{Input methods}
\subsubsection{Input interpretation}
\subsubsection{Representation}

\section{Choise of techniques}

\section{Implementation}

\section{Evaluation}

\section{Conclusion}

\section{Further work}


Cherlin ~\cite{Cherlin:2005:SMF:1090122.1090145} explores techniques for rapid sketch based 3d modeling.

Traditional modeling of three dimensional 
Sketch based modeling is a technique for modeling, where 

\bibliography{thesis}{}
\bibliographystyle{plain}
\end{document}
